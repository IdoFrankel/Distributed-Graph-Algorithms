\documentclass[11pt]{article}

\newcommand{\numpy}{{\tt numpy}}    % tt font for numpy

\topmargin -.5in
\textheight 9in
\oddsidemargin -.25in
\evensidemargin -.25in
\textwidth 7in
\renewcommand{\baselinestretch}{1.25} 

% psudo-code text
\usepackage{algpseudocode}
\usepackage{aligned-overset}

\usepackage{setspace}

\newtheorem{theorem}{Theorem}[section]
\newtheorem{corollary}{Corollary}[theorem]
\newtheorem{lemma}[theorem]{Lemma}


\begin{document}

% ========== Edit your name here
\author{Ido Frankel - 318985108}
\title{Distributed Graph Algorithms HW 2}
\maketitle

\medskip

% ========== Begin answering questions here

\section{Exercise 1}
Few notation which will be useful: (assume that $v \in P_i$)

$\phi(v)$: The maximal vertex index belongs to $P_i$, which $v$ encounter in the the graph.

$N_i(v)$: All of $v$ neighbours in $P_i$

$\lambda(v)$: True/False - indicates whether vertex $v$ is aware of being belong to "big" part (meaning, part with more than $\sqrt{n}$ vertices)

$t-$neighborhood of $v$: are all the vertices with distance at most $t$ from $v$ in the part $v$ belongs to.

\begin{algorithmic}[1]
\State initialization: $\forall i \in [1,\dots, N]$ $\forall v \in P_i$:
\begin{itemize}
    \item $\phi(v) \leftarrow \text{ID}(v)$
    \item $\lambda(v) \leftarrow False$
\end{itemize}

\State Pick a leader (denoted $r$) and build a BFS tree $T$ from it.
\State $n \leftarrow$ compute total number of vertices in $T$

\For{$\sqrt{n} + 1 \text{ iterations  } \forll i \in [1, \dots, N] \; \forall v \in P_i \text{( In parallel)}$ }

\State Send $\phi(v)$ to all $N_i(v)$ (On the original G graph)
\State $\phi(v) \leftarrow \max \{ \{ \phi(v) \} \cup \{  { \phi(v_k) \mid v_k \in N_i(v)} \} \}$

\If{last iteration \& $\phi(v)$ have been updated}
    \State $\lambda(v) \leftarrow $ True
    \State $v$ sends $i$ (the part index) to $r$ via the BFS tree.
\EndIf
\EndFor


\If{$v$ receives an index of a certain part}
    \State sort all indices for propagating, and send one after the other to $r$ (by sending it to its parent in the BFS tree) ($v$ can locally remember which indices it has already sent to the root, and do not send each index twice to root).
    % \State send it to $r$ (by sending it to its parent in the BFS tree) ($v$ can locally remember which indices it has already sent to the root, and do not send each index twice to root).
\EndIf 

\If{$v == r $}
    \State broadcast every index $i$ it receives to all nodes in $T$
\EndIf 

\If{$v$ receives index $i$ (to which it belongs) from root (i.e $P_i$ has more than $\sqrt{n}$ vertices)}
    \State set $H_i  \leftarrow E \backslash E_{P_{i}}$
\Else
    \State set $H_i \leftarrow \emptyset$
\EndIf

\end{algorithmic}

\subsubsection*{Note about the BFS tree and calculation}
I use the BFS algorithm shown in class as a black box, with one minor change. after each node updates within itself its parent node, it also sends an ACK message to the parent, notifying that the edge is indeed part of the BFS tree. (I asked if it is allowed in a reception hour)
In addition the amount of vertices in the trees, can be computed as follows: all the leaves in the tree send a message to their parents in the tree, which then accumulate the number of messages from their children, and send it to their parent ( $+1$ for themselves), until it reaches the root.

\subsection*{Correctness}

\begin{lemma}
\label{t_neighborhood}
(in for loop of row $4$ of the algorithm): after $t$ rounds, $\phi(v)$ is the maximal vertices' ID in its $t-$neighborhood.
\end{lemma}
proof by induction: 

\textbf{Base}: for $t=1$, after a single round, $\phi(v)$ is the maximal ID assignment among all the neighbors of $v$ in $P_i$. meaning it is the maximal ID assignment in $v$'s first-neighborhood.

\textbf{Induction hypothesis}: After $t=k$ rounds $\phi(v)$ is the maximal vertices' ID in its $k-$neighborhood.

\textbf{Induction step}: According to the induction hypothesis, after $k$ rounds all $u \in N_i(v)$ holds: $\phi(u)$ is the maximal ID assignment in their $k$-neighborhood. Because $v$'s $k+1$ neighborhood, consists of all $N_i(v)$'s $k+1$ neighborhood, it holds that after additional rounds $\phi(v)$ is the maximal ID assignment in $v$'s k+1 neighborhood.

\begin{lemma}
\label{discovery_large_diameter_parts}
After $\sqrt{n} + 1$ iterations, in each Part which has a diameter $\omega(\sqrt{n})$, there is a vertex with $\lambda(v) == True$
\end{lemma}
Let $P_i$ be a part with diameter greater than $O(\sqrt{n})$ (meaning, with diameter of at least $\omega(\sqrt{n})$). Assume by contradiction that none of $P_i$'s vertices has modified its $\phi(v)$ value in the last iteration (i.e $\lambda(v) == \text{False}$). Let $s,t \in P_i$ be two vertices which their shortest path length equals the diameter, and let $x \in P_i$ be the vertex with maximal assignment in $P_i$. divide into cases:
\begin{itemize}
    \item $x$ in $\sqrt{n}-$neighborhood of both $t,s$ : then, there is a path $x - \cdots s$ and $x - \cdots - t$, of distance at most $\sqrt{n}$ each. Meaning, there is a path of distance at most $2 \cdot \sqrt{n}$ between $s$ and $t$. contradiction to assumption that the diameter is at least $\omega(n)$.
    \item w.l.o.g $x$ is not in $\sqrt{n}-$neighborhood of t. Let $u$ be a vertex with distance of $\sqrt{n} +1 $ from $x$. (such vertex exists, because $t$ is at distance of at least $\sqrt{n} +1 $, and it is known that each sub-path $(x,u)$ of shortest path between $(x,t)$ is also a shortest path between $(x,u)$). then because $x$ has the maximal assignment in $P_i$, then in iteration $\sqrt{n} +1$, $u$ will update its $\phi(u)$ value. contradiction.
\end{itemize}

\begin{lemma}
\label{dimater_upper_bound_amount_vertices}
The diameter of every connected graph is upper bounded by the number of vertices in the graph
\end{lemma}
 Assume by contradiction that it is false, Let $P_i$ be part with diameter $t$ with $s < t$ vertices. Then there are two vertices which their shortest path between them is $t$, dividing to cases:
\begin{itemize}
    \item The path has more than $s$ unique vertices, which is a contradiction to the fact that $G$ has fewer vertices than that
    \item The path has less or equal to $s$ unique vertices, then according to the pigeonhole theorem there is a vertex that appears twice in the path (meaning there is a cycle), a contradiction to the fact that it is the shortest path. 
\end{itemize}


\begin{lemma}
\label{parts_bound_amount}
There at most $\sqrt{n} -1 $ parts with diameter of $\omega(\sqrt{n})$
\end{lemma}
Each part with diameter of $\omega(\sqrt{n})$ has at least $\omega(\sqrt{n})$ vertices according to lemma \ref{dimater_upper_bound_amount_vertices}.
Let assume by contradiction that there are at least $\sqrt{n}$ parts with more than $\omega(\sqrt{n})$ vertices. then, the total number of vertices in the graph is at greater then $\sqrt{n}  \cdot  (\sqrt{n} +1 )= n + \sqrt{n} > n$. Contrary to the fact that the total number of vertices is $n$.


\begin{lemma}
\label{parts_dimater_D}
each part with diameter larger than $\omega(\sqrt{n})$ holds that the diameter of $G[P_i] \cup H_i$ is $D$
\end{lemma}
According to the algorithm and Lemma \ref{discovery_large_diameter_parts}, if the part has diameter greater than $\omega({\sqrt{n}})$, then $G[P_i] \cup H_i = G$, because each edge that is not part of $H_i$, is part of $G_{P_{i}}$ according to the algorithm. Therefore, the diameter of $G[P_i] \cup H_i$ is $D$.


\begin{lemma}
\label{parts_dimater_sqrt_n}
each part with diameter at most $O(\sqrt{n})$, holds that the diameter of $G[P_i] \cup H_i$ is at most $O(\sqrt{n})$
\end{lemma}
In this case, according to the algorithm $H_i = \emptyset$, then $G[P_i] \cup H_i = G[P_i]$, therefore $G[P_i] \cup H_i$ has diameter of $O(\sqrt{n})$


\begin{lemma}
\label{edges_bound}
Each edge appears in at most $\sqrt{n}$ of the sub-graphs $G[P_i] \cup H_i$
\end{lemma}
According to the lemma \ref{parts_bound_amount}, there are at most $\sqrt{n} -1$ parts with more than $\sqrt{n}$ vertices. According to the algorithm, it means that there are at most $\sqrt{n}-1$ non-empty $H_i$. Furthermore, each edge in the graph can belong to at most one $G[P_i]$, that is, if both of its touching vertices are part of $G[P_i]$, otherwise it will not belong to any $G[P_i]$.
In total, each edge will be part of at most $\sqrt{n}$ of the sub-graphs $G[P_i] \cup H_i$

% \begin{lemma} The algorithm computes $(\sqrt{n}, O(D + \sqrt{n}) )$ shortcut
% \end{lemma}
% Whether the part have dimater of size $O(\sqrt{n})$ or not, according to lemmas \ref{parts_dimater_D} and \ref{parts_dimater_sqrt_n}, its diameter is upper bounded by $D$ and $O(\sqrt{n})$ respectively. This implies that $O(O(\sqrt{n}) +D)=O(\sqrt{n} +D)$ is an upper bound of the diameter of each part. According to lemma \ref{edges_bound}, each edge appears at $\sqrt{n}$ of the subgraphs $G[P_i] \cup H_i$. Therefore, the algorithm successfully computes the desired shortcut.

\begin{lemma}
\label{vertices_notify}
At the end of the algorithm, each node knows the shortcut of the graph.
\end{lemma}
By Lemma \ref{discovery_large_diameter_parts} and the algorithm (row 12,13), each part with diameter greater than $O(\sqrt{n})$, has at least one vertex which sends its part index to the root. Then the root, will broadcast that index to all vertices in the graph. meaning, each vertex knows $H_i$ of each part. Hence it can calculate the shortcut of its part. 


\begin{lemma} The algorithm computes $(\sqrt{n}, O(D + \sqrt{n}) )$ shortcut
\end{lemma}
Whether the part have dimater of size $O(\sqrt{n})$ or not, according to lemmas \ref{parts_dimater_D} and \ref{parts_dimater_sqrt_n}, its diameter is upper bounded by $D$ and $O(\sqrt{n})$ respectively. This implies that $O(O(\sqrt{n}) +D)=O(\sqrt{n} +D)$ is an upper bound of the diameter of each part. According to lemma \ref{edges_bound}, each edge appears at $\sqrt{n}$ of the subgraphs $G[P_i] \cup H_i$, and according to Lemma \ref{vertices_notify}, each part knows its shortcut. Therefore, the algorithm successfully computes the desired shortcut.


$\newline$
Note: The calculation of each $\phi(v)$ is performed in parallel among different parts. Since each part $P_i$ is disjoint in node with all other parts, and because each node send the information regarding its $\phi$ value only to its $N_i(v)$ neighborhood, which are edges with both edges in $P_i$, i.e., each message as described stays within the component. Therefore, the calculation of different parts can be achieved in parallel.


\begin{lemma}
each vertex knows when the algorithm ends
\end{lemma}
after the for loop ends, each vertex knows that the root will send it at most $\sqrt{n}$ indices of $P_i$'s (according to lemma \ref{parts_bound_amount}). It takes at most at most $2 \cdot (D + \sqrt{n})$ for each vertex to send the message to the root, and then the root to notify all the vertices in the graph (according to lemma \ref{root_brodcast_pipeline} and \ref{TODO} Hench  $2 \cdot (D + \sqrt{n})$ iterations, after the for loop ends, the vertices knows it will not get any message from the root, and the computation ends.

\subsection*{Complexity}

\subsubsection*{claims}

\begin{lemma}
\label{root_brodcast_pipeline}
 It takes $D + \sqrt{n}$ iterations for $r$ to send at most $\sqrt{n}$ messages (i.e indices) to all the nodes in the graph
\end{lemma}
(Pipelining, was proven in previous exercise, Question 2, sub-question 1).


% \begin{lemma}
%  It takes $D + \sqrt{n}$ iterations for each $v \text{ with } \lambda(v) == True$ to send its $P_i$'s $i$ index to $r$
% \end{lemma}
% According to Lemma \ref{parts_bound_amount}, there are at most $\sqrt{n}$ unique parts' indices which will be sent to $r$, and according to the algorithm, each node does not sends the same message twice (i.e send only message which it has not seen before), As a results, each vertex sends at most $\sqrt{n}$ messages to $r$. We will prove by induction, that it takes $i + \sqrt{n}$ rounds for vertex at distance $i$ from $v$ to sends it part's index to $r$

% textbf{Base}: for $i=1$, $v$ is a neighbour of $r$, hence it takes him $\sqrt{n}$ rounds to sends at most $\sqrt{n}$ unique messages. (one message per round). (in the worst case, his part index will be sent last).

% \textbf{Induction hypothesis}: assume correctness for $i$, proof for $i+1$

% \textbf{Induction step}: According to the induction hypothesis, it takes $i + \sqrt{n}$ rounds for each vertex at distance $i$ from $r$ to sends its part's index to the root. as we have shown in the base case, it takes $\sqrt{n}$ rounds to send all the unique messages to the parent in the 

\begin{lemma}
 It takes $D + \sqrt{n}$ iterations for $r$ to receive all the indices of big parts.
\end{lemma}
In previous HW (question 2, sub-question 2), we have shown that if each vertex has $k_v$ messages, then it takes $O(D+\sum{k_v})$ rounds for the root to know all the k pieces of information.
According to Lemma \ref{parts_bound_amount}, there are at most $\sqrt{n}$ parts which we will denote as "big" parts, in which we want to send their index to the root. Hence each vertex can send at most $\sqrt{n}$ unique messages (meaning, it is impossible for two nodes, to send different $\sqrt{n}$ messages). We will show by induction that sending $k$ unique messages from $v$ (which is at distance $i$) to $r$ takes $i + k$ rounds.

\textbf{Base}: for $i=1$, $v$ is a neighbour of $r$, hence it takes him $k$ rounds to send $k$ unique messages. (one message per round). (in the worst case, his part index will be sent last).

\textbf{Induction hypothesis}: assume correctness for $i$, proof for $i+1$

\textbf{Induction step}: According to the induction hypothesis, it takes $i + k$ rounds for each vertex at distance $i$ from $r$ to sends its part's index to the root. hence, after the first message arrives to $v$'s parent in the BFS. (and according to the pipeline principle prooved in precious exercise Q2.1) it takes $i + k$ rounds for all the message to arrive to $r$ from $v$'s parent. hence it takes $i+ 1 + k$ rounds for $v$ to send all messages to the root.

\subsubsection*{computations}

\begin{enumerate}
    \item The process of picking a leader and the construction of a BFS tree, as taught in class is $O(D)$. It takes $O(D)$ rounds for each message that $v$ sends to reach the furthest node in the tree (since the diameter is D). 
    \item The calculation of number of vertices in each BFS tree is $O(D)$, as we have shown in the previous exercise. (each leaf sends ACK to $r$, and each vertex accumulates the amount sent by its children, and passes it to $r$, (including itself)).
   \item It takes $\sqrt{n} +1$ iterations for each part with diameter greater than $O(\sqrt{n})$, to have at least one vertex with $\lambda(v) == True$ by Lemma \ref{discovery_large_diameter_parts}.
   \item It takes $D + \sqrt{n}$ iterations for each $v \text{ with } \lambda(v) == True$ to send its $P_i$'s $i$ index to $r$ by Lemma \ref{vertex_sends_root} (Pipeline).
   \item It takes $D + \sqrt{n}$ iterations for $r$ to send at most $\sqrt{n}$ messages (i.e indices) to all the nodes in the graph by Lemma \ref{root_brodcast_pipeline}
\end{enumerate}
Total: $O(D + \sqrt{n})$ rounds

%\pagebreak%
\section{Exercise 2}
\begin{algorithmic}[1]
\State initialize $C_i = {v_i}$ (each vertex as a separate component)
    \While{There are outgoing edges from component $C$}
    \State $\forall C, \forall v\in C$, $v$ sends to all its neighbours an outgoing edge (which belongs to $H$) from $C$.
    \State All vertices send all their neighbors in $C$, one outgoing edge from $C$ (via the network graph)
    \State compute maximal matching on outgoing edges that belongs to $H$
    \State Merge components adjacent to the edges of the matching
    \State Merge neighboring unmatched components that do not have adjacent edge in the matching
    \State in each merge of components $C_1, C_2$, the vertices adjacent to the selected edge (which caused the merge), learns the minimal ID of the components and then sends it to all vertices in $C_1, C_2$
    \EndWhile
\end{algorithmic}

\subsection*{Correctness}

\begin{lemma}
After O(1) rounds, all vertices of $C$ learn the outgoing edge from $C$ 
\end{lemma}
Each vertex sends one message to each neighbor, which in turn locally computes one edge (that is, an edge $(u,v)$ with decreasing id of $u,v$) from the input it receives and then sends one message to all its neighbors in $C$ with that edge.
Because the diameter of the network graph is 2, then in $O(1)$ rounds, each vertex in $C$ knows which outgoing edge was picked.

\begin{lemma}
calculation of outgoing edge from each component is done in parallel
\end{lemma}
Each vertex in $C$ sends one message to all of its neighbors, later, each vertex sends a single message to any neighboring vertex in $C$. i.e., different components received the messages on different edges. And only one message is sent on each edge (in each direction). 

\begin{lemma}
\label{message_transfer_2_rounds}
A vertex can send a message to all of the component's vertices in 2 rounds.
\end{lemma}
Similarly to the calculation of the outgoing edge, the vertex sends all its neighbors the message, and then the neighbors themselves transmit this message to their neighbors in $C$ (each edge transfers at most 1 message in each direction).

\begin{lemma}
After the merge it takes $O(1)$ rounds for all the vertices in the merged component to learn the new ID.
\end{lemma}
According to the lemma \ref{message_transfer_2_rounds}, the adjacent vertices of the merged edge can send the new chosen ID in 2 rounds.

\begin{lemma}
The Algorithm computes the connected components of $G_H$
\end{lemma}
At each iteration, the algorithm merges components that have a connected edge of $H$. meaning ,we merge only connected components under $G_H$. In addition, the algorithm stops only when there are no outgoing edges from all components. Therefore, the algorithm stops when each component contains all connected vertices in $G$ under $H$

\subsection*{Complexity}
\begin{itemize}
    \item In each iteration, the number of connected components is reduced by at least half, as each component is merged with at least one component. Therefore, there are at most $\log{n}$ iterations.
    \item In each iteration, the calculation of maximal matching takes $O(\log^{*}{n})$
    \item The broadcast of the new component ID, to all vertices of the merged component takes $O(1)$
    \item The calculation of the selected outgoing edge takes $O(1)$
\end{itemize}
The algorithm takes $O(\log{n} \cdot \log^{*}{n})$ rounds.

\section{Exercise 3}

\subsection{}
Assume by contradiction that there exists an induce graph $G[V_i]$ that has a vertex of degree greater than $t$. According to the definition of the induced graph, it means $v$ have more than $t$ vertices with same color as $v$ in $G[V_i]$. Meaning $\phi$ is not a valid $t$-defective $k$-coloring function.

\subsection{}

\subsubsection*{Correctness of $t + \frac{\Delta}{p}-$defective $p^2$ coloring}

\begin{lemma}
\label{psi_p2_coloring}
$\psi$ is $p^2$ coloring function.
\end{lemma}
$\psi$ consists as a concatenation of $\psi_1, \psi_2$, each $p-$coloring function. Therefore, $u$ outputs a color that can be decoded by $2 \cdot \log{p}=\log{p^2}$ bits. There are $2^{\log{p^2}}=p^2$ options. Therefore, $\psi$ is the $p^2$ coloring function. 

\begin{lemma}
\label{psi_defective_3}
$\psi$ is $t+\frac{\Delta}{p}$-defective
\end{lemma}
First, we will present few notations which will be useful. \\
Denote $N_{\phi(v)}(v)= \left\{ u \mid u \in N(v) \land \phi(v) = \phi(u) \right\}$. \\
Denote $A = \{ u \in N_{\phi(v)}(v) \mid \text{u has choose } \psi_1 \}$, let $k_1 = |A|$ \\
Denote $B = \{ u \in N_{\phi(v)}(v) \mid \text{u has choose } \psi_2 \}$, let $k_2 = |B|$ \\
Denote $C = \overline{A \cup B}$ (Note $0 \le |C| \le t$), let $k_3 = |C| $

According to the inclusion-exclusion principle, the number of vertices of $N_{\phi(v)}(v)$ which have not choose either $\psi_1$ or $\psi_2$ is $k_3 = t - |A \cup B| = t - |A| - |B| + |A \cap B| = t - (k_1 + k_2) + |A \cap B|$.
Hence, According to the algorithm and the pigeon hole theorem
\begin{itemize}
    \item the total number of vertices which can share $\psi_1(v)$ is $\frac{|S(v)| + k_1}{p}$
    \item the total number of vertices which can share $\psi_2(v)$ is $\frac{|G(v)| + k_2}{p}$
    \item the total number of vertices which have not choose any $\psi_1, \psi_2$ is $k_3$
\end{itemize}
lets define $\psi(v) = \psi_1(v), \psi_2(v)$. In the worst case, any vertex $u$ which share $\psi_1$ color with $v$, also shares its $\psi_2$ color, similarly any vertex $x$ which share $\psi_2$ color with $v$, also shares its $\psi_1$ color. In addition in the worst case each vertex $u \in C$ can choose $\psi(v)$ as their color. \\
To conclude, in the worst
\begin{doublespacing}
\newline
$\frac{|S(v)| + k_1}{p} + \frac{|G(v)| + k_2}{p} + k_3 =$
\\[5mm]
$\frac{|S(v)| + |G(v)|}{p} + \frac{k_1 + k_2}{p} + (t - k_1 - k_2 +  |A \cap B|) =$ 
\\[5mm] 
$\frac{|S(v)| + |G(v) +t }{p} - \frac{t}{p} + \frac{k_1 + k_2}{p} + (t - k_1 - k_2 +  |A \cap B|) =$
\\[5mm]
$\frac{\Delta}{p} + t + \frac{k_1 + k_2 - p \cdot k_1 - p \cdot k_2 - t + p  \cdot  |A \cap B|}{p}  = $
\\[5mm]
$\frac{\Delta}{p} + t + \frac{k_1(1-p) + k_2(1-p) - t + p  \cdot  |A \cap B|}{p} \underbrace{\le}_{|A \cap B| \le \min({k_1, k_2}), \text{ w.l.o.g } |A \cap B| \le k_1}$
\\[5mm]  
$\frac{\Delta}{p} + t + \frac{k_1(1-p) + k_2(1-p) - t + p \cdot k_1}{p} =  \frac{\Delta}{p} + t + \frac{1}{p} \cdot (\underbrace{k_1 - t}_{\le 0} + \underbrace{k_2(1-p)}_{\le 0}) \le  \boxed{\frac{\Delta}{p} + t}$
\end{doublespacing}


Meaning, that in the worst case $v$ shares its $\psi$ color with at most $\frac{\Delta}{p} + t$ of its neighbors


\begin{lemma}
$\psi$ is $t+\frac{\Delta}{p}$-defective $p^2$-coloring.
\end{lemma}
According to Lemma \ref{psi_defective_3} and Lemma \ref{psi_p2_coloring}

\subsubsection*{Complexity claims}

\begin{lemma}
\label{psi_1_rounds}
It takes $O(\phi(v) -1)$ rounds for $v$ to choose $\psi_1{v}$
\end{lemma}
proof by induction: 

\textbf{Base}: for $\phi(v)=1$ it takes at most 0 rounds to choose $\psi_1(v)$
because there is no vertex $u$ with $\phi(u) < \phi(v)$, then by definition $S(v)=\emptyset$. Therefore, $v$ can choose $\psi_1(v)$ without the need to wait for the response of other vertices.

\textbf{Induction hypothesis}: for $\psi(v)=l-1$ it takes at most $l-2$ rounds to choose $\psi_1(v)$

\textbf{Induction step}: According to the induction hypothesis, all nodes with $\phi(u)=l-1$ can choose $\psi_1(u)$ in at most $l-2$ rounds. Let $v$ be a vertex with color $\phi(v)=l$. then, after at most $l-2$ rounds, all $S(v)$ compute their $\psi_1$ color. It takes another round, for them to send $v$ their $\psi_1$ color. Therefore, after at most $l-1$ rounds, $v$ can locally compute $\psi_1$

\begin{lemma}
\label{psi_2_rounds}
It takes $O(k -\phi(v) -1 )$ rounds for $v$ to choose $\psi_2(v)$
\end{lemma}
similar proof to \ref{psi_1_rounds} (descending induction) as such, I will skip it. (Please do not kill me :) )

\begin{lemma}
\label{psi_rounds}
It takes $O(k)$ rounds for $v$ to choose $\psi(v)$
\end{lemma}
By Lemmas \ref{psi_1_rounds} and \ref{psi_2_rounds}, it takes $O(\phi(v))$, $O(k- \phi(v))$  rounds to calculate $\psi_1(v)$ and $\psi_2(v)$ respectively. Because $v$ needs both $\psi_1, \psi_2$ in order to determine $\psi$, it takes \\
$\max \left\{ O(\phi(v)), O(k- \phi(v)) \right\}$
% $\max \left\{ {O(\phi(v)), O(k- \phi(v)) }\right\}$
rounds, which is equivalent to $O(\phi(v) + k- \phi(v))=O(k)$ rounds.


\subsubsection*{Complexity calculation}
\begin{itemize}
    \item Sending a message of length $\log{k}$ (where $k \le n$) to all neighbors takes a single round
    \item computing $S(v), G(v)$ which can be done locally takes 0 rounds.
    \item Choosing $\psi_1(v), \psi_2(v)$ takes $O(k)$ rounds according Lemma \ref{psi_rounds}
    \item Step 5 is done locally, takes at most a single round.
\end{itemize}
Total: $O(k)$


\subsection{}

\begin{algorithmic}[1]
\State Use $\mathcal{A}_1$ to compute a $\frac{\Delta}{\log{\Delta}}-$defective $(\log^2{\Delta})$ coloring.
\For{$i=0 \: ; \: i < \log^2(\Delta) \: ; \: i++ $}
    \State Apply $\mathcal{A}_2$ on $G[V_i]$ ($G[v_i]$ consists same color-vertices as defined in exercise 3.1) in order to compute a legal $O(\frac{\Delta}{\log{\Delta}})$ coloring.
\EndFor

\State Set unique color for each induced graph $G[V_i]$

% \State Use the iterate color reduction algorithm from the lecture.

\State Apply the parallel color reduction algorithm from the tutorial.

\end{algorithmic}

\subsubsection*{Correctness}

\begin{lemma}
Calculation of algorithm $\mathcal{A}_2$ on each induced graph takes $T_2(\frac{\Delta}{\log{\Delta}})$ rounds
\end{lemma}
Algorithm $\mathcal{A}_1$ computes a $\frac{\Delta}{\log{\Delta}}$ defective coloring function, according to exercise $3.1$ it implies that each induced graph $G[V_i]$ has maximal degree of $\frac{\Delta}{\log{\Delta}}$. As such algorithm $\mathcal{A}_2$ computes on each induced graph a valid $O(\frac{\Delta}{\log{\Delta}})$ in $T_2(\frac{\Delta}{\log{\Delta}})$ rounds according to its definition.

\begin{lemma}
\label{A_2_parallel}
The calculation of $\mathcal{A}_2$ on each induced graph can be done in parallel
\end{lemma}
Because each induced graph includes only vertices of same color, it means, there is not a vertex which is part of two $G[V_i], G[V_j]$. Hence, every induced graph is disjoint with all others. That is, the application of $\mathcal{A}_2$ can be done in parallel and during the execution of the algorithm, no message is sent from one induced graph to another.

\begin{lemma}
In step $5$ we defined a $0-$defective $O(\Delta \cdot \log{\Delta})$ coloring for $G$
\end{lemma}
After application of algorithm $\mathcal{A}_2$, each induced graph has valid $O(\frac{\Delta}{\log{\Delta}})$ coloring function by definition (denote $\varphi_i)$. In step 5, we set a unique color scheme for each of the $\log^2{\Delta}$ induced graphs, it means every two induced graphs do not share any color. Hence we can construct a  $0-$defective $O(\frac{\Delta}{\log{\Delta}}  \cdot  \log^2{\Delta})
=O(\Delta  \cdot  \log{\Delta})$-coloring function of $G$ as follows: $\forall v \in G$ if $v \in V_i$, set $\Phi(v) = \varphi_i(v)$. 
\begin{lemma}
\label{parallel_reduction_tutorial}
The parallel color reduction algorithm from the tutorial, which preforms on graph with maximal degree $\Delta$ with given $O(\Delta  \cdot  \log{\Delta})$ coloring function $\Phi$ takes $O(\Delta  \cdot  \log({\log{\Delta}}))$
\end{lemma}
$\Phi$ is $0-$defective $O(\Delta \cdot \log{\Delta})$ coloring, $G$ has maximal degree of $\Delta$. According to the algorithm presented in the tutorial, given $\Phi$ we can compute $\Delta+1$ coloring in 
$O(\Delta  \cdot  \log{\frac{C}{\Delta}})$ where $C$ is the number of colors of $\Phi$, hence the computation takes $O(\Delta  \cdot  \log{\frac{\Delta \cdot \log{\Delta}}{\Delta}})=O(\Delta  \cdot  \log({\log{\Delta}}))
$


\subsubsection*{Complexity}
\begin{itemize}
    \item Step 1 takes $T_1(\Delta)$ rounds
    \item According to lemma \ref{A_2_parallel} Step 2 is calculated on each induced graph in parallel. Therefore, the for loop is preformed in parallel and takes in total $T_2(\frac{\Delta}{\log{\Delta}})$ rounds. 
    \item The Algorithm of Step 6 takes $O(\Delta\log(\log{{\Delta}}))$ rounds by Lemma \ref{parallel_reduction_tutorial}
\end{itemize}

Total complexity: $T_1(\Delta) + T_2(\frac{\Delta}{\log{\Delta}}) + O(\Delta  \cdot  \log({\log{\Delta}}))$ 


\end{document}


