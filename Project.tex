\documentclass[11pt]{article}
\usepackage{fullpage}
\usepackage[utf8]{inputenc}
\usepackage{algorithm}
\usepackage{algorithmic}
\usepackage{amsfonts}
\usepackage{graphicx}
\usepackage{amsthm}
\usepackage{mathrsfs}
\usepackage{amssymb}
\usepackage{hyperref}
\usepackage{amsmath}
\usepackage{cite}
\usepackage{xcolor}
\newtheorem{theorem}{Theorem}
\newtheorem{lemma}{Lemma}
\newtheorem{definition}{Definition}
\newtheorem{claim}{Claim}
\newtheorem{corollary}{Corollary}
\newtheorem{observation}{Observation}
\newtheorem{remark}{Remark}
\newtheorem{oq}{Open Question}


\begin{document}
\title{Final Project - Distributed Graph Algorithms - Spring 2018\\
On the paper: "Paper Title", by "Paper Authors"
}
\author{Name\footnote{Email, Student Number} \and Name\footnote{Email, Student Number}
}
\date{date}
	\maketitle
\section{Summary}	
This section should contain a 2-4 page summary of the paper that was assigned to you. The length of this section is not an evaluation criteria and so maximizing it should not be your aim.
The summary should be understandable to a reader that is familiar with what we study in the course \emph{without the need to read the paper}. The summary should focus on the ideas that are given in the paper, rather than explicit notations and proofs. It must be written in your own words. Please keep in mind that copying parts of the assigned paper is strictly forbidden.
\section{Related Work}
This section should cover work that is related to the assigned paper. It should cover work that was chronologically done before this paper, and also work that was done after it (use online sources, such as Google Scholar).
\section{New Results}
This section should contain a description of a new result. Think about the task of the assigned paper in \emph{other models} of distributed computing. Think about \emph{other tasks} in the model of the assigned paper. Think about removing or adding \emph{assumptions} to the task or the model. Think about subcases (a family of graphs rather than a general graph, etc.). Your project certainly does not have to be a publishable result (though there have been such cases in the past), but it should show your understanding of the different aspects of distributed computing.

%\bibliographystyle{alpha}
%\bibliography{bib-filename}


\end{document} 