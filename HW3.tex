\documentclass[11pt]{article}

\newcommand{\numpy}{{\tt numpy}}    % tt font for numpy

\topmargin -.5in
\textheight 9in
\oddsidemargin -.25in
\evensidemargin -.25in
\textwidth 7in
\renewcommand{\baselinestretch}{1.25} 

% psudo-code text
\usepackage{algorithm}
\usepackage{algpseudocode}
\usepackage{aligned-overset}

\usepackage{fontspec}
% \usepackage[noend]{algpseudocode}

\newcommand*\Let[2]{\State #1 $\gets$ #2}
\algrenewcommand\algorithmicrequire{\textbf{Precondition:}}
\algrenewcommand\algorithmicensure{\textbf{Postcondition:}}

\usepackage{xfrac}


\usepackage{setspace}

\newtheorem{theorem}{Theorem}[section]
\newtheorem{corollary}{Corollary}[theorem]
\newtheorem{lemma}[theorem]{Lemma}


\begin{document}

% ========== Edit your name here
\author{Ido Frankel - 318985108}
\title{Distributed Graph Algorithms HW 3}
\maketitle

\medskip

% ========== Begin answering questions here

\section{Exercise 1}

\subsection{Algorithm}

\begin{algorithm}
  \caption{$(\tilde{\Delta} + 1)$-coloring in Beeping model}
  \begin{algorithmic}[1]
    \Statex
    \Function{$(\tilde{\Delta} + 1)$-coloring}{}
        \Let{v.deactivate}{False} \Comment{$\forall v \in G$}
        \For{$i=1, \dots$}
            \For{$c=1, \dots, \tilde{\Delta}+1$}
                \State each active vertex $v$ sends a beep with probability \sfrac{1}{$i$}
                \If{$v$ beeped, and doesn't receive a beep from its neighbours}
                    \Let{v.color}{c}
                    \Let{v.deactivate}{True}
                    \State notify neighbours about the color \Comment{each color iter. is divided to 2 slots}
                    \State \Return{c}            
                \EndIf
            \EndFor
        \EndFor
    \EndFunction
  \end{algorithmic}
\end{algorithm}

\begin{theorem}
\label{vertex_colored_itertaion_i}
Each vertex is being colored in iteration $i$ w.p $1 - \frac{\Delta}{i}$
\end{theorem}
Let $v$ be a node, with degree $d$, Let $X_{v_i}$ be an indicator variable that indicates if the neighbour $u_i$ of $v$ has picked same color as $v$. Let $X_v=\sum_{i=1}^{d}X_{v_i}$, then $Pr[X_{v_i}=1]=\sfrac{1}{i}$ and $E(X_v)= d \cdot \frac{1}{i} $. By Markov's inequality, $Pr[X_v \ge 1] \le \frac{d}{i} \le \frac{\Delta}{i}$. Hence $Pr[X_v = 0] \ge 1 - \frac{\Delta}{i}$, which is the probability that none of $v$'s neighbours has been colored with same color as $v$.






\begin{theorem}
\label{legal_coloring_i}
The algorithm finishes in iteration $i$ w.p $1- \frac{n \cdot \Delta}{i}$
\end{theorem}
Using theorem \ref{vertex_colored_itertaion_i} and Union bound, we get \[
Pr[\exists \: v, \text{which was not colored by iteration } i]=Pr[\bigcup (X_v \ge 1)] \underset {\text{U.B}}{\underbrace{\le}} \sum Pr[X_v \ge 1]= \] \[ n \cdot  Pr[X_v \ge 1] \le 
\frac{n \cdot \Delta}{i} \]
. Hence, the complementary probability, which is $Pr[\text{all } v, \text{were colored by iteration } i] \ge 1- \frac{n \cdot \Delta}{i}$


\begin{theorem}
The algorithm computes legal coloring w.p 1
\end{theorem}
By theorem \ref{legal_coloring_i}, the algorithm finishes computing legal coloring in iteration $i$ w.p at least $1- \frac{n \cdot \Delta}{i}$. Hence as $i \to \infty$, the probability of the algorithm to produces a valid coloring is \[\underset{i\to\infty}{\lim} P(\text{all vertices has been legally colored)}= \underset{i\to\infty}{\lim} 1- \frac{n \cdot \Delta}{i} =  1 \]



\subsection{Lower Bound}
Assume by contradiction there is an algorithm \(\mathcal{A}\) which computes $\Delta+1$ coloring in $\Delta -1$ rounds on Clique $G$ with $n$ vertices (thus the degree is $\Delta=n-1$). Denote $G'=(V',E')$ with $V'=V \cup \{v'\}$, and $E' = \{(s,t) \mid s,t \in V'\}$. $v'$ is a copy of specific node in $G$, which holds that in each iteration it does not beep alone (meaning, it either hasn't beeped, or if it beeps, there exists another vertex which beeps in same iteration).
Denote \(\mathcal{A}\), \(\mathcal{A'}\) as the execution of the algorithm on $G,G'$ respectively.


\begin{lemma}
\label{single_beep_different_input}
$v_1, v_2$ receive different input (under \(\mathcal{A}\)) in iteration $i$ iff only single vertex in the graph have beeped in iteration $i$, and it's either $v_1$ or $v_2$ 
\end{lemma}
$\leftarrow$: w.l.o.g $v_1$ have beeped (send $1$), and $v_2$ haven't (send $0$), Because no other vertex have beeped, than $v_1$ received $0$ as input, while $v_2$ received $1$, resulting with $v_1,v_2$ receiving different inputs.
\newline
$\rightarrow$: $v_1, v_2$ received different input, w.l.o.g $v_1$ received $1$ and $v_0$ received $1$, than according to the model, $v_1$ has a neighbour which beeped, Because $v_2$ haven't received beep from any of its neighbours, its mean $v_2$ have beeped, and its the only vertex which have beeped.

\begin{lemma}
\label{single_beep_j_iterations}
In $j$ iterations, there are at most $j$ vertices which have sent beep alone in their iteration (meaning that in the iteration which they beeped, there weren't any other vertex which beeped).
\end{lemma}
In each iteration, there are two options:
\begin{enumerate}
    \item a single vertex have beeped, while other haven't
    \item there are several vertices which have beeped, or no one have beeped.
\end{enumerate}
Hence, the first option can occur at most $j$ times (one per iteration).

\begin{lemma}
\label{v_exists}
such $v'$ exists.
\end{lemma}
By lemma \ref{single_beep_j_iterations}, in $\Delta-1 = n-2$ iterations there are at most $n-2$ vertices which have sent beep alone (under \(\mathcal{A}\)). By pigeonhole principle there exists a vertex which hasn't sent a beep alone during all the execution of \(\mathcal{A}\)

\begin{lemma}
The execution (input and output) of all vertices in $V$ is the same under \(\mathcal{A'}\) as in  \(\mathcal{A'}\), and the execution of $v'$ equals to $v$ in every iteration.
\end{lemma}
proof by induction

\textbf{Base}: In iteration $i=1,$ every vertex does not have any previous knowledge, hence the output of every $u\ne v'$ vertex under \(\mathcal{A}\) and \(\mathcal{A'}\) is the same. and the output of $v'$ equals $v$ (because one is duplication of the other)


\textbf{Induction hypothesis}: Assume that for every iteration $i$, $\forall u \ne v'$ has same input and output under \(\mathcal{A'}\) as in \(\mathcal{A}\), and $v'$ has the same input and output as $v$

\textbf{Induction step}: proof for iteration $i+1$,
\begin{itemize}
    \item \underline{$v$ has beeped in iteration $i+1$ under \(\mathcal{A}\)}: By induction hypothesis, all vertices has the same execution (input and output) up until the current iterations, hence $v \text{ will}^{\star}$ also beep under \(\mathcal{A'}\). By lemma \ref{v_exists}, $v$ does not beep alone, hence there exists $u\in (V - \{v\})$ which also beeps in the current iterations, meaning the input of all vertices (in particular $v,v'$) is the same. 
      \item \underline{$v$ has not beeped in iteration $i+1$ under \(\mathcal{A}\)}: than $v'$ hasn't beeped as well, as it is a duplication of $v$ and has same execution history as $v$. In addition because $G'$ is clique, the input of all vertices is the same as it was in \(\mathcal{A}\).
\end{itemize}
Meaning, the execution of every vertex $u \in V$ under \(\mathcal{A}\) and \(\mathcal{A'}\) is the same. and the execution of $v'$ equal to the execution of $v$ under \(\mathcal{A'}\).
Meaning, in every iteration $v,v'$ produce same output (because $v'$ is duplication of $v$) and receive the same input as shown, Thus \(\mathcal{A}\) and \(\mathcal{A'}\) produce same color for $v,v'$ at the end of the algorithm, which is contradiction to valid coloring of clique.

Note($\star$): If vertex has same execution (input and output) of some algorithm until iteration $i$ in both \(\mathcal{A}\) and \(\mathcal{A'}\)., it means that even if the algorithm is randomized (and the next phase might be one of several options), there exists a path in the decision tree, which leads to the same next execution as it did under \(\mathcal{A}\). (The "will" - refers to  \(\mathcal{A}\), as the desire flow, which is possible).

\newpage
\section{Exercise 2}

\subsection{Warm-up}

Note:
At first, the question was finding a solution with the CONGEST model, which was later changed to LOCAL. The correctness of the algorithm is the same, the only difference is that in CONGEST I showed $O(n \cdot k)$ iterations while in local I showed $O(k \cdot \log{n})$
I kept the complexity derivations of CONGEST, and added a seperate for LOCAL.

\begin{algorithm}
  \caption{Finding ($k, k\log{n}$)-ruling set for $U \subseteq V$}
  \begin{algorithmic}[1]
    \Statex
    \Function{Ruling}{$U, k, k\log{n}$}
        \If{$U$ only $1-$bit set} \Comment{each vertex in U is one bit}
            \State \Return{vertex with bit 0}
        \EndIf
        
        \State{$U_0 \overset{\triangle}{=} \{ u \in U \mid u\text{'s first bit is } 0 \} $} \Comment{$\forall u \in U_0,U_1:$ omitting the first ID's bit}
        \State{$U_1 \overset{\triangle}{=} \{ u \in U \mid u\text{'s first bit is } 1 \} $}
      
        \State{$S_0 = Ruling(U_0, k,k(\log{n}-1))$} \Comment{$S_0, S_1$ computation is in parallel}
        \State{$S_1 = Ruling(U_1, k,k(\log{n}-1))$}

        \State{$S = S_0 \cup \{s_1 \in S_1 \mid \forall s_0 \in S_0:$ dist$(s_1, s_0) \ge k\}$}
        
        \State \Return{$S$}
    \EndFunction
  \end{algorithmic}
\end{algorithm}


\subsubsection*{Correctness}

\begin{theorem}
\label{rulling_set_i}

The algorithm calculates valid $(k, k \cdot i)-$ruling set of $U$ consists vertices with $i$ bits.
\end{theorem}
Proof by induction on the ID's length (number of bits), we will prove that after the $i$'th stage of the recursion the algorithm computes a valid $(k, k \cdot i)-$ ruling set $S^{|i|}$, of $U^{|i|}$. (when $U$, consist of vertices with $i$ bits)

\textbf{Base}: for $U^{|i|}$ ($U$ contains 1 bit vertices) is a set containing at most 2 vertices (1 bit, each), thus $S$ is clearly a valid $(k, k)-$ ruling set of $U^{|1|}$ (the set in the bottom of the recursion tree).


\textbf{Induction hypothesis}: Assume that for $U^{|i-1|}$ (an $i-1$ bits set) S is a valid $(k, k \cdot (i-1))-$ ruling set. 

\textbf{Induction step}: According to the induction hypothesis, $S^{|i-1|}_{0}, S^{|i-1|}_{1}$ are a valid $(k, k*(i-1))-$ruling sets of $U^{|i-1|}_{0}, U^{|i-1|}_{1}$ respectively. According to line 9 of the algorithm, S is a union of $S_0$ and vertices $s_1 \in S_1$ which are at distance at least $k$ from $S_0$.
\begin{itemize}
    \item \underline{$\forall s_1, s_2 \in S \: \text{dist}(s_1,s_2) \ge k$:}
        \begin{itemize}
            \item if $s_1, s_2 \in S^{|i-1|}_{0}$ or $s_1, s_2 \in S^{|i-1|}_{1}$, the claim holds immediately, since they are valid ruling-sets.
            \item w.l.o.g $s_0 \in S^{|i-1|}_{0}$, $s_1 \in S^{|i-1|}_{1}$, then by definition of $S$, it consists only vertices in $S^{|i-1|}_{1}$ which are at distance at least of $k$ from every vertex in $S^{|i-1|}_{0}$
        \end{itemize}
    Hence the separation claim of $S$ is satisfied.
    
    \item \underline{$\forall v\in U^{|i|}, \exists s \in S^{|i|}$ which holds $\text{dist}(s,v)$ $\le k \cdot i$}:
    
    Note $U^{|i|}=U^{|i|}_0 \cup U^{|i|}_1$, and $S^{|i-1|}_0$ is a valid $(k, k\cdot(i-1))-$ruling set.
    
    if $v \in U^{|i-1|}_0$, It holds that $\exists s_0 \in S^{|i-1|}_0$
    s.t \boxed{\text{dist}(v,S^{|i-1|}_{0}) = \text{dist}(v, s_0) \le k \cdot (i-1) \le k \cdot i}.
    
    Otherwise $v \in  U^{|i-1|}_1$, hence there exists $s_1 \in S^{|i-1|}_1$ s.t $\text{dist}(v, s_1) \le k \cdot i$.
    If $s_1 \in S$, we done, otherwise because $s_1 \in U^{|i|}$, it holds there exists $s''$ s.t $\text{distance}(s_1, s'') \le k$, Hence according to the triangle inequality we receive (for $s'' \in S \cap S^{|i-1|}_{1})$: 
    
    \boxed{\text{dist}(v,S^{|i-1|}_1) \le \text{dist}(v,s'') + \text{dist}(s'',s_1) \le k \cdot (i-1) + k = k \cdot i}
    
    Hence \boxed{dist(v, S) \le \min\{\text{dist}(v,S^{|i-1|}_0), \text{dist}(v,S^{|i-1|}_1)\} \le k \cdot i}
    
\end{itemize}

\begin{corollary}
\label{rulling_set_logn}
The algorithm computes valid  $(k, k\cdot(\log{n}))-$ruling set
\end{corollary}
By theorem \ref{rulling_set_i}, for set $U$ consists of $i$ bits vertices, the algorithm computes an $(k, k \cdot i)-$ruling set. Thus for $U$ which consists of $n$ vertices (each vertex ID's length is $\log{n}$), it computes a valid $(k, k \cdot \log{n})-$ruling set


\subsubsection*{Complexity}
Note: The recursion depth begins with depth 0

\begin{corollary}
At depth $i$ of the recursion, Calculating $S$ takes $O(k* 2^{i})$  iterations in CONGEST
\label{calculating_S_1_iteration_i}
\end{corollary}

According to the algorithm, calculating $S$ requires to know which vertices of $S_1$ are at distance at least $k$ from $S_1$, it can be achieved in $k$ times \#number-of-groups iterations at the $i$'th stage of the recursion as follow:
$\forall s \in S_0$, s sends  "$k$" to all its neighbours, each vertex sends the maximal number it received minus 1. (and stop sending when reaching negative number) (meaning, if neighbor received multiple messages from same group in a single round, it takes the maximal among each group and broadcast it, in addition it will broadcast only messages with higher $k$ values then it previously sent for each group). if $s_1 \in S_1$ received a message from any $s_0$, it knows that it should not be in $S$, as it distance from $s_0$ is less than $k$.

At stage $i$ of the recursion, the number of different $S'_1$ groups at the recursion tree is $2^{i}$ (e.g. at 0-depth, there are only $2^{0}=1$ different $S'_1$ groups, to calculate distance from). In the worst case, vertex $v$ might send $k$ messages from each group. Thus $O(k \cdot 2^{i})$ 


\begin{theorem}
\label{Complexity_2_1}
Complexity is $O(k \cdot n)$ rounds in CONGEST
\end{theorem}
There are $\log{n}$ recursion levels in the recursion tree, each level takes
$O(k \cdot 2^{i-1})$ rounds by corollary \ref{calculating_S_1_iteration_i}.
$\sum_{i=0}^{\log{n}-1} k \cdot 2^{i} = O(k \cdot n)$ rounds

\begin{corollary}
At depth $i$ of the recursion, Calculating $S$ takes $O(k)$  iterations in LOCAL
\label{calculating_S_1_iteration_i_LOCAL}
\end{corollary}
According to the algorithm, calculating $S$ requires to know which vertices of $S_1$ are at distance at least $k$ from $S_1$, it can be achieved in $k$ iterations at the $i$'th stage of the recursion as follow:
$\forall s \in S_0$, s sends  "$k$" to all its neighbours, each vertex sends the maximal number it received minus 1. (and stop sending when reaching negative number) (meaning, if neighbor received multiple messages from same group in a single round, it takes the maximal among each group and broadcast it, in addition it will broadcast only messages with higher $k$ values then it previously sent for each group). if $s_1 \in S_1$ received a message from any $s_0$, it knows that it should not be in $S$, as it distance from $s_0$ is less than $k$.

At stage $i$ of the recursion, under LOCAL model, each vertex can send in a single round all the messages it received. Hence each message will pass through $k$ nodes, and then will be discarded. Hence $O(k)$ iterations.

\begin{theorem}
\label{Complexity_2_1_local}
Complexity is $O(k \cdot \log{n})$ rounds in LOCAL
\end{theorem}
There are $\log{n}$ recursion levels in the recursion tree, each level takes $O(k)$ rounds by corollary \ref{calculating_S_1_iteration_i_LOCAL}. Hence $O(k \cdot \log{n})$ rounds.

\newpage
\subsection{Ruling Set from Coloring}
Note: this question is under LOCAL model.

Roadmap: we will divide $V$ to $c^{\frac{1}{k}}$ different group, where each group contains vertices with color taken from a predefined subset of $[c]$. Later we will use recursion (similarly to previous subsection) in order to compute a (2,k) ruling set of each group $U_i$, and finishes with sequential passing over all $U_i$, and taking vertices of $U_i$ only if its distance from previous $U_{j< i}$ is at least 2.
For simplicity we will assume the given $c$-coloring spans the range $[0, c-1]$.
Denote the coloring function as $\phi$.

\begin{algorithm}
  \caption{Finding (2,k)-Ruling Set}
  \begin{algorithmic}[1]
    \Statex
    \Function{RulingSet}{$V$, $[c_{begin}, c_{end}]$ (color indices)}
        \Let{$\lambda$}{$\lfloor(c_{end} - c_{begin})^{\frac{1}{k}}\rfloor$}
        \If {$\lambda == 1$} \Comment{only single color in set}
            \Let{$U$}{$V$}
        \Else
            \Let{$colors_i$}{$[i \cdot \lambda, ((i+1) \cdot \lambda ) - 1]$} 
            
            \Let{$V_i$}{$ \{ v \in V \mid \phi(v) \in colors_{i} \} $}
            \Let{$U_i$}{$RulingSet(V_i, colors_i)$} \Comment{In parallel for every group}
            
            \Let{$U$}{$U_0$}
            \For{$j=1, \dots, j \le \lambda -1$}
                \For{$v \in U_j$} \Comment{Done in parallel for every vertices within same group}
                    \If{$\not \exists u \in (N(v) \cap U)$}
                        \Let{$U$}{$U \cup \{ v\}$}
                    \EndIf
                \EndFor
            \EndFor
            
        \EndIf
        
    \EndFunction
  \end{algorithmic}
\end{algorithm}

\subsubsection*{Correctness}

\begin{lemma}
\label{U_RULING_SET_correctness}
The algorithm calculates valid (2, k)−ruling set of V
\end{lemma}
The Algorithm correctness is heavily similar to previous subsection. proof by induction on the recursion depth, 
%when the algorithm, returns from the $i$ call of the recursion, the result is $(2, k \cdot {\frac{c}{i}}^{\frac{1}{k}})$
% start of induciton =================================================================================

\textbf{Base}: for $i = 1$ (bottom recursion step) ($V$ is monochromatic vertex set). Because $c$ is valid coloring, any two vertices $x,y \in V$ satisfies $dist(x,y) \ge 2$, hence $U$ is valid $(2,1)-$ruling set

\textbf{Induction hypothesis}: Assume that for the last $i$ recursion stage (from bottom) returns a valid (2,$i$) ruling set.

\textbf{Induction step}: According to the induction hypothesis, $\forall j\in[1, \lambda^{k-1}] U^{|i+1|}_{j}$ are all valid $(2, i)-$ruling sets of $V^{i+1}_{j}$ respectively. According to line 13 of the algorithm, U is a union of $U_0$ and vertices from all $U_{ \le j}$ s.t $u_{j_k} \in U_j$ is at distance at least 2 from $U$.
\begin{itemize}
    \item \underline{$\forall u_1, u_2 \in U \: \text{dist}(u_1,u_2) \ge 2$:}
        \begin{itemize}
            \item if $u_1, u_2 \in U^{|i-1|}_{j}$ (for some $j) $, the claim holds immediately, since $U^{|i-1|}_{j}$ is valid ruling set.
            \item if $u_0 \in S^{|i-1|}_{j}$, $u_1 \in S^{|i-1|}_{k}$ (w.l.o.g $j < k$), then by definition of $U$, it consists only vertices of $U^{|i-1|}_{k}$ which are at distance at least of 2 from every vertex in $U$ (which by the time $U_k$ is treated, $U$ might consists vertices from $U_j$).
        \end{itemize}
    Hence the separation claim of $U$ is satisfied.
    
    \item \underline{$\forall v\in V^{|i|}, \exists u \in U^{|i|}$ which holds $\text{dist}(v,u)$ $\le i$ }:
    
    Note $V^{|i|}= \bigcup_{j} V^{|i-1|}_{j}$, and $\forall j\in[1, \lambda^{k-1}] U^{|i-1|}_j$ is a valid $(2,i-1)-$ruling set.
    if $v \in V^{|i-1|}_0$, It holds that $\exists u_{0_l} \in U^{|i-1|}_0$
    s.t \boxed{\text{dist}(v,U^{|i-1|}_{j}) = \text{dist}(v, u_{0_l}) \le i}.
    
    Otherwise $v \in  V^{|i-1|}_{l \ge 1}$, hence there exists $u_{l_m} \in U^{|i-1|}_l$ s.t $\text{dist}(v, u_{l_m}) \le i-1$.
    If $u_{l_m} \in U$, we done, otherwise because $u_{l_m} \in V^{|i|}$, it holds there exists $u''\in U$ s.t $\text{distance}(u_{l_m}, u'') \le 1$ ($u''$ and $u_{l_m}$ must be neighbours, otherwise, $u_{l_m}$ would have been in $U$) , Hence according to the triangle inequality we receive: 
    
    \boxed{\text{dist}(v,U^{|i-1|}_j) \le \text{dist}(v,u'') + \text{dist}(u'',u_{l_m}) \le i-1 + 1 = i}
    
    Hence \boxed{dist(v, U) \le \min_{j}\{\text{dist}(v,U^{|i-1|}_j)\} \le i}
    
\end{itemize}

\begin{corollary}
\label{rulling_set_2_k}
The algorithm computes valid $(2, k)-$ruling set
\end{corollary}
By theorem \ref{U_RULING_SET_correctness}, for set $V$ with $c$-coloring, the algorithm computes an $(2, i)-$ruling set (where $i$ is the depth of the recursion). Thus for $U$ which is divided to $\lambda = c^{\frac{1}{k}}$ groups in each recursion step, has at most $\log_{\lambda}{c}$ recursion stages, which is upper bounded by $k$ (as I will show in lemma \ref{U_RULING_SET_complexity}. Thus the algorithm computes valid $(2, k)-$ruling set.


\subsubsection*{Complexity}
\begin{lemma}
\label{U_RULING_SET_complexity}
The algorithm takes $O(k \cdot c^{\frac{1}{k}})$ iterations to compute $U$
\end{lemma}
In each recursion step, we divide $V$ to $\lambda$ different groups, which is later followed by $\lambda -1$ iterations in order to compute unify valid ruling set of $v$ of the given recursion step. Hence we get the following formula $T(c) = T(\frac{c}{\lambda}) + \lambda - 1$. In every iterations, the size of $V$ is decreased by $\lambda$ (divided to $\lambda$ groups), hence the recursion depth is at most $\log_{\lambda}{c}$. It holds $\lambda = c^{\frac{1}{k}} \rightarrow \log_{\lambda}{\lambda} = \frac{1}{k} \cdot \log_{\lambda}{c} \rightarrow k = \log_{\lambda}{c}$. 
Thus $ T(c) = T(\frac{c}{\lambda}) + \lambda - 1 \le \log_{\lambda}{c} \cdot (\lambda -1)  \le \log_{\lambda}{c} \cdot \lambda \le k \cdot \lambda \le \frac{1}{k} \cdot c^{\frac{1}{k}}$ iterations.


\newpage
\section{Exercise 3}

\subsection{Handling Insertions}
After the local-ration $2-$approximation on $G$ has finished, we insert an edge $\{u,v\}$. We can obtain $2-$approximation $G_1$ as follows:
buying $z_{u,v}=\min\{w'(u), w'(v)\}$ (where $w'$, is the weight function after the 2-approximation has finished on $G$).

\begin{algorithm}
  \caption{Finding local-ratio 2-approximation after insertion}
  \begin{algorithmic}[1]
    \Statex
    \Function{local-ratio 2-approximation with insertion}{$G$, $e=\{u,v\}$}
        \State run local-ration 2-approximation on $G$ \Comment{Denote VC as $U$}
          \If{$s \notin \{u,v\}$}
            \State \Return{True iff $s \in U$}
        \EndIf
        
        \If{$s \in U$}
            \State \Return{s in VC}
        \EndIf
        \State $z_e=\min\{w'(u),w'(v)\}$ \Comment{$w'$ is remaining weight values after the 2-approximation on $G$}
        \Let{$w'(u)$}{$w'(u) - z_e$}
        \Let{$w'(v)$}{$w'(v) - z_e$}
        \Let{$U$}{$U \cup \{t \in \{u,v\} \mid w'(t) =0$ \}}
        \State \Return{s in VC if w'(s) == 0}
    \EndFunction
  \end{algorithmic}
\end{algorithm}

After the local-ratio 2-approximation algorithm runs on $G$, It takes $O(1)$ rounds to compute which vertex among $v,u$ has smaller weight, and enters the minimal to the cover (if both have same weight, enters both).
Denote $\text{OPT}(G)$ as the optimal VC of G.

\begin{corollary}
\label{u_wight_lower_than_neighbours_sum}
$\forall u \in \text{OPT}(G), w(u) \le \underset{s \in N(u)}{\sum} w(s)$ 
\end{corollary}
Assume by contradiction that it is not correct, then 
$\text{OPT}(G') = (\text{OPT}(G) - \{u\} ) \cup \{ s \mid s \in N(u) \}$, we have the following.

\begin{align*}
\boxed{w(\text{OPT}(G))} = \underset{s \in OPT(G)}{\sum} w(s) = ( \underset{s \in OPT(G) - \{u\} }{\sum} w(s) ) + w(u) \boxed{>} \\ > ( \underset{s \in OPT(G) - \{u\} }{\sum} w(s) ) + \underset{t \in N(u) }{\sum} w(t)= \boxed{w(OPT(G')}
\end{align*}

Thus, $w(\text{OPT}(G)) > w(\text{OPT}(G'))$ and OPT$(G')$ is a valid set cover, since all edges are covered either by their original vertex, or if an edge was covered by $u$, it now covers by its neighbor. All of the above leads to the contradiction to the optimally of $OPT(G)$


\begin{theorem}
The procedure computes a valid 2-approximation VC of $G_1$
\end{theorem}
Let $U$ be a valid 2-approximation VC of $G$, and let $\{u,v\}$ be the edge inserted to $G$, creating $G_1$. Divide into cases:
\begin{itemize}
    \item \underline{Either $u,v \in U_G$}: $U_G$ is a valid 2-approximation VC of $G$, since $e=\{u,v\}$ is covered by either $u,v$ it is also 2 approximation VC of $G_1$ as 
    $w(U_{G_1})=w(U_G) \le w(OPT(G) \le w(OPT(G_1))$ (as the optimally VC of $G_1$ is also VC of $G$)
    \item \underline{both $u,v \notin U_G$}:
        \begin{itemize}
            \item \underline{both $u,v \notin OPT(G)$}: w.l.o.g $w(u) \le w(v)$, hence according to the algorithm $u \in U_{G_1}$ (as it must be valid VC), thus by by corollary \ref{uv_notin_UG_OPTG} $w(OPT(G)) \le w(OPT(G_1)) - w(u)$ holds. Meaning
            \begin{align*}
            w(U_{G_1}) = w(U_G) + 2\cdot \min\{w(u), w(v)\} = w(U_G) + 2\cdot w(u) \le 2 \cdot w(OPT(G) + 2\cdot w(u) \\ \le 2 \cdot w(OPT(G_1)) - 2 \cdot w(u) + 2\cdot w(u) = 2 \cdot w(OPT(G_1))
            \end{align*}

            Hence $U_{G_1}$ is a valid 2-approximation VC of $G_1$
            
            \item \underline{w.l.o.g $u \in OPT(G)$}: by corollary \ref{u_wight_lower_than_neighbours_sum} it holds $w(u) \le \underset{s \in N(u)}{\sum} w(s)$. if $w(u)$ is lower than of of its neighbours, than it would have been part of $OPT(G)$ after the algorithm finish running on $G$, otherwise $w(u)$ is greater than all of its neighbours, hence in each itertation, $w(u)$ will be decreased by the $w(s)$ for $s\in N(u)$, hence by corollary  \ref{u_wight_lower_than_neighbours_sum}, when 2-approximation algorithm on $G$ finishes $w(u)=0$. Contradiction that $u \notin U_G$
        \end{itemize}
\end{itemize}


\begin{corollary}
If $u,v \notin U_G \land u,v \notin OPT_G$ then $w(OPT(G)) \le w(OPT(G_1)) - \min\{w(u), w(v)\}$
\label{uv_notin_UG_OPTG}
\end{corollary}
Assume by contradiction $w(OPT(G)) > w(OPT(G_1)) - \min\{w(u), w(v)\}$. w.l.o.g $w(u) \le w(v)$. then by taking $u$ to $OPT({G_1})$, and then choosing the optimal VC on the remaining graph, we will receive an optimal VC of $G$, (which by adding $u$ is also valid (not necessary optimal) VC of $G_1$)

According to the assumption $w(OPT(G)) > w(OPT({G_1})) - \min\{w(u), w(v)\}$, meaning the valid VC of ${G_1}$ (without $u$), which is also valid VC of $G$, has lower weight than $OPT(G)$. contradiction to $OPT(G)$ optimally.


\subsubsection*{Complexity}
It takes $O(1)$ rounds to obtain the 2-approximation $G_1$ (given 2 approximation on $G$), as describe above (calculating which of $u,v$ has lower remaining weight).

[for completeness]
In total (without previous knowledge about the 2-approximation of $G$ it takes $O(V)$ iterations, as in each iteration, we decrease at least one vertex to 0-weight. hence after $O(V)$ iterations there aren't any vertices in the remaining graph).

\newpage
\subsection{Handling Deletions}
Let $e=(u,v)$ be the edge which was deleted from the graph, and let $S$ be the 2-approximation VC of the original $G$ graph.
Note: w.l.o.g assuming $v \in S$, if also $u \in S$, assume w.l.o.g $w(v) \le w(u)$

\begin{itemize}
    \item Denote $G'=(V',E')$ as $V'=V, E'=E-\{e\}$.
    \item Denote $\mathcal{A}$ as the local-ratio algorithm procedures on $G$, and $\mathcal{A'}$ be an identical procedures steps on $G'$.
    \item for $e=(x,y)$, Denote $Z_e=\min\{w_G(x), w_G(x)\}$, the discarded weight (in both $\mathcal{A, A'}$) as it was when $e$ was chosen in $\mathcal{A}$
\end{itemize}
Note that every weight reduction during $\mathcal{A'}$ is determined by the weighs of G, (and not by the weights of $G'$).

\begin{algorithm}
  \caption{Finding local-ratio 2-approximation after deletion}
  \begin{algorithmic}[1]
    \Function{local-ratio 2-approximation with deletion}{$G$, $S$, $e=\{u,v\}$}
        \Let{U}{S}
        \If{e was not chosen by 2-approx algorithm on $G$}
            \State \Return{S}
        \EndIf
        
        \State $w(u) = w(u) + Z_e$ \Comment{At the end of approx-execution on $G$}
        \State $w(v) = w(v) + Z_e$ \Comment{$Z_e = \min\{w_G(u), w_G(v)\}$}
                
        \State $u$ waits while $v$ finishes next two steps \Comment{u waits for 2 iterations}
         \For{$x \in N(v)$}
                \State $x$ sends $v$, its weight at the end of the algorithm.
            \EndFor
        \State $v$ simulates local-ratio of its neighbours, and send them their respective updated weight.
        \If{$x \in N(v) \cup \{v\}$ and $w(x) == 0$}
            \Let{$U$}{$U \cup \{x\}$}
        \EndIf
        
        \If{$u \in S$}
            \For{$x \in N(u)$}
                \State $x$ sends $u$, its weight at the end of the algorithm.
            \EndFor
            \State $u$ simulates local-ratio for all touching edges, and sends its neighbours their respective weight.
        
            \For{$x \in N(u) \cup \{u\}$}
                \If{$w(x) == 0$}
                    \Let{$U$}{$U \cup \{x\}$}
                \EndIf
            \EndFor
        \EndIf
    \EndFunction
  \end{algorithmic}
\end{algorithm}

\begin{lemma}
\label{w_g_less_w_g_tag}
$\forall x \in V, w_G(x) \le w_{G'}(x)$ for every iterations in both \( \mathcal{A}, \mathcal{A'}\)
\end{lemma}
proof by induction on iterations (note that if $e$ was selected during the execution of $\mathcal{A}$, then we will add an idle step in $\mathcal{A'}$ for consistency.
Denote the edge which was handle during the $i$'th iteration as $e_i$

\textbf{Base}: In iteration $i=1,$:
\begin{itemize}
    \item \underline{if $e_i = e$}: we haven't changed the weight of edges in $G'$ and in $G$ we have decrease the weights of $u,v$ by $Z_e$, other vertices left unchanged, hence $w_G(x) \le w_{G'}(x)$
    \item \underline{if $e_i=(x,y) \ne e$}: we have decreased both graph by $\min\{w_G(x),w_G(y)\}$
\end{itemize}
Hence for every $u \in V: w_G(u) \le w_{G'}(u)$


\textbf{Induction hypothesis}: Assume correctness for iterations until $i$

\textbf{Induction step}: proof for iteration $i+1$,
\begin{itemize}
    \item \underline{if $e_i = e$}:
        \begin{itemize}
            \item $for x \ne u,v$ we haven't change the weight, then by induction assumptions $w_G(x) \le w_{G'}(x)$
            \item for $u,v$ by induction hypothesis $w_G(u) \le w_{G'}(u)$ and $w_G(v) \le w_{G'}(v)$, thus by reducing $z_e$ in $G$ and non reducing in $G'$ the inequality still holds.
        \end{itemize}
    \item \underline{if $e_i=(x,y) \ne e$}: in both $G,G'$ we reduce the weight by  $\min\{w_G(x),w_G(y)\}$, hence the inequality holds.
\end{itemize}


\begin{lemma}
\label{lemma_2}
every edge $e\in E'$ s.t $u,v \notin e$ is covered by $\mathcal{A'}$
\end{lemma}
$\mathcal{A}$ on $G$ is a local-ratio algorithm, hence according to its correctness, by the end of the algorithm, it covers every edge $(x,y) \in E$, and in particular every edge $(x,y) \in E'$ s.t $x,y \ne u,v$. Meaning either $x,y$ (or both) has $w_G(x), w_G(y)=0$. By lemma \ref{w_g_less_w_g_tag} we get that at the end of execution $\mathcal{A}, \mathcal{A'}$ every vertex $s \ne u,v$ holds $w_G(s)=w_{G'}(s)$, thus by lemma \ref{w_g_less_w_g_tag} it satisfies $w_{G'}(s) = 0$ 

\begin{lemma}
\label{lemma_3}
Every weight reduction step is valid local-ratio step
\end{lemma}
As noted before, every weight reduction step is done by the weight of the corresponding iteration and vertex weight in $G$, as such, for $e=(x,u), (x,v)$, we might not reduce any of touching vertices to zero weight in a single iterations. nevertheless it is still a local-ratio step (according to lecture 9, slide 17), but its takes more iterations to reduce a vertex weight to zero, and then taking at least one to the cover

\begin{lemma}
\label{A_TAG_AS_WHILE_LOCAL_RATIO}
$\mathcal{A'}$ is valid local-ratio algorithm
\end{lemma}
Under $\mathcal{A}$ execution on $e=(x,y) \in E'$, we reduce $z_e$ from both endpoints. By lemma \ref{w_g_less_w_g_tag}, it holds $w_G'(x) \ge w_G(x)$, hence every iteration in $\mathcal{A'}$ satisfies $z_e = \min\{w_G(x), w_G(x)\} \le \min\{w_{G'}(x), w_{G'}(x)\}$, means every iteration of $\mathcal{A'}$ is local-ratio step.

\begin{lemma}
\label{lemma_5}
At the end of $\mathcal{A, A'}, \forall x \ne \{u,v\} \: w_G(x)=w_{G'}(x), \forall y\in\{u,v\} \: w_{G'}(y)=w_G(y) + Z_e$
\end{lemma}
Let $x\ne u,v$, than except the choice of $e$, both algorithms preforms the same on all nodes, meaning every weight reduction which is preform in $G$ also occurs in $G'$, thus $ w_G(x)=w_{G'}(x)$

for $y=u,v$: in case $e$ was not elected by $\mathcal{A}$, then $w(y)$ was not reduced in $\mathcal{A}$ as well as in $\mathcal{A'}$ (meaning, $Z_e = 0$), hence $ w_G(y)=w_{G'}(y)$, if it was elected by $\mathcal{A}$, then it was reduced by $Z_e$, but not in $\mathcal{A'}$ (because the edge does not exists), Thus $w_{G'}(y)=w_G(y) + Z_e$


\begin{lemma}
\label{A_TAG_VALID_VC}
The algorithm produces valid VC of $G'$
\end{lemma}
By lemma \ref{lemma_5} it holds that
\begin{itemize}
    \item every edge $(x,y) = e\in E' w_G(x)=w_{G'}(x), w_G(y)=w_{G'}(y)$, by the correctness of the original algorithm on $G$, it produces valid VC, thus either $x$ or $y$ satisfies $w_G=0$, hence $w_{G'} = 0$, meaning the edge is covered. 
    \item $e= (t,v)$: (w.l.o.g $v \in S$), it holds  $w_{G'}(v)=w_G(v) + Z_e$, after the steps of the original algorithm on $G'$, $v$ compute local-ratio steps on its neighbours (line 12), thus either $w_{G'}(v)=0$ or $w_{G'}(t) = 0$, hence the edge is covered.
    \item $e = (t,u)$:
        \begin{itemize}
            \item $u \in S$: same procedure is applied as $v$, hence $e$ is covered
            \item $u \notin S$: following the correctness of the original $\mathcal{A}$ on $G$, all of $u$'s neighbours have been added to $S$ (otherwise, it was not valid cover), hence because in $G'$, we only increased the weight of $u$, and because we preform same operations as in $\mathcal{A}$, all of $u$'s neighbours are also in the cover. (before step 20)
        \end{itemize}
\end{itemize}
To conclude every edge is covered by $\mathcal{A'}$.

\begin{theorem}
The algorithm produces valid 2-approximation VC of $G'$
\end{theorem}
\begin{itemize}
    \item \underline{$e=(u,v)$ was not selected by the original 2-approximation algorithm on $G$}: then the resulting VC is an optimal VC of $G'$, since we could have removed $e$ from $G$ and getting same local-ration steps, hence the correctness is achieved by the local-ratio 2 algorithm, and our algorithm, stops at line 2.
    \item \underline{$e=(u,v)$ was selected}: according to lemma \ref{lemma_2} and \ref{lemma_5}, every vertex $x \ne u,v$ holds $w_{G}(x)=w_{G'}(x)$ and $y \in \{u,v\}$ holds $w_{G'}(y)=w_G(y)+Z_e$
    \begin{itemize}
        \item if $v \in S, u\notin S$ then after simulating local-ration step on $v$'s neighbours and updates $U$ accordingly (line 12). then by the correctness of the algorithm on $G$ (lemma \ref{A_TAG_VALID_VC}) , because $u \notin S$, then all of its remaining edges are covered by its neighbours. In addition by lemma \ref{lemma_3} and \ref{A_TAG_AS_WHILE_LOCAL_RATIO} the operation is valid local-ratio step, hence in every iteration $U$ is at most twice as the optimal solution.
        \item otherwise $u \in S$ (reminder: we assumed $v \in S$ in the first place), and preforms local ratio steps on its neighbours as well (line 20), and updating $U$ accordingly. Hence by lemma \ref{A_TAG_VALID_VC}, and because the operation is valid local-ration step, $U$ is a 2-approximation VC of $G'$.
    \end{itemize}
\end{itemize}


\subsubsection*{Complexity}
Every step of the algorithm is either local computations or requires $O(1)$ rounds. (The steps which involves sending messages to its neighbours, is preformed in parallel, as there is a single message which is being sent on every edge). Thus the algorithm takes \underline{$O(1)$ iterations}.

\end{document}